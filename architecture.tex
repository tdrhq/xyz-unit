\documentclass{article}
\title{Architecture of xyz-Unit testing framework}
\author{Arnold Noronha}
\begin{document}

\maketitle
\begin{abstract}
This document defines an architecture of two extensions (\emph{yUnit} and \emph{zUnit})to current
unit testing frameworks.
\end{abstract}

\section{Introduction}

While there are several hundreds of Unit testing frameworks, they
suffer from many real drawbacks, that make them difficult to use. The
first that this document tries to solve is that of separating the
\emph{fixture}, i.e., the data, from the actual test scenario. This
framework will be called \emph{yUnit}. 

The second issue, is of ``unit
testing'' within the natural environment of the component. For
instance, in a single-threaded, event-driven application, a component
that relies on network feedback, may not have actions that return
independently. Thus we need to have an event loop running for
performing the unit test. Some developers might wish to call this
``integration testing'', but we'll not distinguish between the
two. The \emph{zUnit} API we define below intends to simplify the
process of writing unit tests in such environments, and may not be
suitable for applications that are heavily thread dependent.

This document attempts to define a programming interface that we
would can be consistently replicated in different programming
languages. An exact specification is provided only for C.


\section{Definitions and the Testing Paradigm}

We follow the following paradigm of unit testing: for each test,
\begin{enumerate}
\item Set up the ``state'' of the program or environment under which
  the module has to be tested, and
\item Do the test
\end{enumerate}

We want to make the process of setting up the ``state'' convenient and
easy. For that purpose, we define the following:

\begin{itemize}
\item {\bf State.} The state of the program, or the environment under
  which the software runs. The definition of state could be program
  specific, and will depend on the intuition of the testers and
  developers, but it is important to define the state early
  on. For example, for a C program, the state could be defined with a
  high level of granularity: as the exact memory dump. However, since
  this is of less practical use, we will typically define the state as
  the set of all data structs, and the pointers between them. 
  
  In a CGI web application, a state might be the combination of the
  database contents, and the session variables (cookies).
\item {\bf Fixture.} A state of a program is typically too complicated
  to specify for each test. We define a fixture, as a pre-defined
  state, data, or configuration settings before the application
  runs. Again, this might vary for, say, web applications.

  In database-driven web applications, there could be an easy
  distinction between State and Fixture. The Fixture could be just the
  database contents, while the state could be the database contents
  along with the contents of the Session (the current user logged in,
  etc.).

  In C programs, the fixture could be the set of configuration files,
  along with the data the program operates on. For example, in a feed
  reader, the fixture could be the URLs of feeds, and all cached
  data. However the user might perform actions once the program starts
  that do not change the fixture, but will change the state of the
  program (e.g., view a feed).

\end{itemize}

  Fixtures makes defining the state convenient. For example, we can
  now change our testing paradigm to the following:

\begin{enumerate}
\item A tester (writing a specific test) chooses one of the fixtures to
  base her test on. 
\item She then builds up the state of the system from that fixture by
  running commands/actions on the intermediate states,
\item and finally she tests the action she's interested in.
\end{enumerate}

Most developers and testers using the xUnit model of testing actually
use the above paradigm, although there's no clear distinction between
step 1 and step 2.

\section{yUnit}

For the yUnit framework, we note that in the testing paradigm,
building up the state of the system might still be hard. What if the
tester wants to write the same test over different states of the
system? This means that a tester needs to design her tests very
well. In many case, this design can be redundantly similar.

Step 2 and Step 3 can vary depending on the fixture chosen in Step
1. If we could write Steps 2 and 3 independent of the fixture, then
it's just a matter of running the same test over various fixtures to
get a very high degree of testing on the component.

The idea then, would be to not hardcode the fixture-specific constants
into the test code (steps 2 and 3). One way xUnit does this, is to use
data providers. A data provider would typically be part of the test
code itself, that passes a variety of values to the test code.

However, we'd like to be able to support the following paradigm:
fixtures are hard to generate, and we'll want to keep this is as a
more ``clerical'' work, or to leave it to the user-community in an
open source project. A tester should write only Step 2 and Step 3,
without knowledge of the fixture. (The tester may set constraints on
the fixture.) Any fixture-specific values are not hardcoded. Once the
test code is written, a tester may take the test case,
run it on an existing fixture (or create a new fixture), provide
values for all the test fixture-specific values, and verifies that the
test produces the right answer (which can be specific to the fixture)
for the test. At this point, the tester or user may choose to enter
this fixture, test code combination into a database: i.e., the
database consists of all the fixture-specific values the FP provided. 

The combination of the test code (or simply, \emph{test}) and the
fixture and the fixture variables constitutes a \emph{scenario}. A
\emph{scenario provider} would specifically refer to a tester or even
a user who combines the fixture and test, with the glue fixture variables.

While the test can typically be part of the actual development
code repository, the scenarios should be specified externally. E.g., a user
might use a scenario that depends on his passwords, and so he wouldn't
want to share this scenario. However, users may sync their databases
of scenarios via push and pull mechanisms analogous to distributed
version control systems such as Git. It is conceivable that the
project may control a central database of scenarios, that all other
testers and FPs push to. 

The format of the database, is independent of the programming language
the test or the program is written in.

\subsection{Naming}

Since each test, fixture, and scenario are synced from multiple users,
each of them require a unique name. This document suggests the use of
Java inverted domain name convention to keep them unique, however
projects are allowed to use any naming they prefer.

\subsection{Database specification}

The database creates the following tables:

\begin{itemize}
\item \emph{List of test names.} This database may be linked to a
specific version of the code, and so may not support all the tests
that another database for the same code may support.
\item \emph{List of fixtures}, a key-value pair, listing for each fixture name the
contents of the fixture. It is hard to specify exactly how
the fixture looks like, since different applications may use different
notions as to how to store their data or configuration values. The
test plan would include creating a \verb|test_serialize| and
\verb|test_unserialize| to serialize the entire state of the fixture into a
single string. (Careful here, keeping track of fixture versions,
anyone?)
\item \emph{A list of all scenarios}. For each scenario, we should be
  able to store the name of the scenario, the fixture it uses, the
  test it uses, and a table of $[key,value]$ string pairs indicating
  each of the test variables.
\end{itemize}

In order to unify and simplify the above in implementation, we allow
the database to be the form of a tree, with each node being a key
string and the leaf node being a value for the key. Each leaf is
specified by the names of the keys along the path. The root node will
be an empty string and will be skipped in specifying the key
value. The first level in the tree will contain exactly three keys:
``tests'', ``fixtures'', and ``scenarios''.

The document does not specify the implementation of such a structure.

\pagebreak

\section{Headers}
\subsection{Database Headers}
\begin{itemize}
\item {\bf NAME}

   yunit/db.h --- low level manipulation of the yUnit key database

\item {\bf SYNOPSIS}

   \verb|#include <yunit/db.h>|

\item {\bf DESCRIPTION}

   The \verb|<yunit/db.h>| header shall provide functions to manipulate and
   access the scenario and test database.

   The \verb|<yunit/db.h>| header shall declare the \verb|_YUnitDb|
   struct and typedef it as follows

\begin{verbatim}
typedef struct _YUnitDb YUnitDb;
\end{verbatim}

   The \verb|<yunit/db.h>| header shall declare the \verb|_YUnitDbKey|
   struct and typedef it as follows:
\begin{verbatim}
typedef struct _YUnitDbKey YUnitDbKey;
\end{verbatim}

   The following shall be defined as functions. If any function
   returns a \verb|char*|, then that return value needs to be freed
   using \verb|free|.
\begin{verbatim}
YUnitDb*    yunit_db_new (void);
YUnitDb*    yunit_db_load (const char *filename);
void        yunit_db_save (const YUnitDb *db,
                           const char *filename);
YUnitDbKey* yunit_db_get_root (YUnitDb *db);
YUnitDbKey* yunit_db_key_create (YUnitDb *root,
                                 const char *key0,
                                 ...);
void        yunit_db_key_free (YUnitDbKey *key);
char*       yunit_db_get_value_k (YUnitDbKey *key);
char*       yunit_db_get_value (const char* key0,
                                ...);
void        yunit_db_set_value_k (YUnitDbKey *key,
                                  const char *value);
void        yunit_db_set_value (const char* value,
                                const char *key0,
                                ...);
\end{verbatim}

\end{itemize}
\pagebreak
\subsection{Test Headers}

\begin{itemize}
\item {\bf NAME}

yunit/test.h --- calls used during the execution of a test

\item {\bf SYNOPSIS}

\verb|#include <yunit/test.h>|

\item {\bf DESCRIPTION}

The \verb|<yunit/test.h>| header shall provide routines that are used
during the execution of a Scenario. A \emph{key} in this Description
will be a string that refers to a key within the scenario description,
and should not be confused with the database key.

The \verb|<yunit/test.h>| header shall provide the following function
that should be called before any of the other functions in this
description are called. This function may set up key-value pairs
based on information from the environment variables.

\begin{verbatim}
void yunit_test_init (void);
\end{verbatim}
The \verb|<yunit/test.h>| header shall provide the following low level
routines to manipulate the current run of a scenario.

\begin{verbatim}
void    yunit_test_set_key (const char *key, const char *value);
char    *yunit_test_get_key_ (const char *key);
\end{verbatim}

The \verb|<yunit/test.h>| shall provide the following functions as the
standard way for a test to access the scenario data. These functions
may block for user input, however the means of interaction with the
user is not specified by this document.

\begin{verbatim}
char *yunit_test_get_key     (const char *key,
                              const char *description);
int  yunit_test_get_key_int  (const char *key,
                              const char *description);
int  yunit_test_get_key_bool (const char *key,
                              const char *description);
\end{verbatim}

The \verb|<yunit/test.h>| header shall provide the following interface
for allowing the developer to specify his own mechanism for user input
when required:

\begin{verbatim}
typedef char* YUnitTestUserInput (const char *key, const char
*description);
void   yunit_test_set_user_input (YUnitTestUserInput *func);
\end{verbatim}


\end{itemize}


\end{document}


